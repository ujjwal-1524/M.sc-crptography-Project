\documentclass[12pt,a4paper]{article}
\usepackage[utf8]{inputenc}
\usepackage[T1]{fontenc}
\usepackage{amsmath}
\usepackage{amssymb}
\usepackage{makeidx}
\usepackage{graphicx}
\author{Ujjwal Kumar Garg}
\begin{document}
		\pagenumbering{roman}
	
		\begin{center}
		
		
		\vspace{10pt}
		\textbf{ {Major Project Phase-Two  \\  RSA vs ECC : A COMPARTIVE ANALYSIS\\
				
		}} \\\textbf{\begin{figure}[h]
				\centering
				\includegraphics[width=0.4\linewidth]{"../../../Downloads/Web capture_28-6-2023_175856_www.bing.com"}
			\end{figure}\\Department of Mathematics\\ GURU GHASIDAS VISHWAVIDYALAYA, \\ BILASPUR, (C.G.), INDIA.}\\  \vfill
		{\large	 \vspace{10pt} \vspace{11pt}
			Submitted in partial fulfilment of the requirements of the degree of  \\  \vspace{10pt} 
			{\Large {\textbf{Master of Science} }}\\ \vspace{5pt}
			in \\ 
			\textbf{Mathematics}\\ \vspace{13pt}
			By\\ \vspace{5pt}
			\textbf{Ujjwal Kumar Garg}\\  \vspace{3pt}
			Roll No: 21075153\\Enrolment No. GGV/21/05735 \\ \vspace{30pt}
			under the supervision of\\ \vspace{10pt}
			\textbf{Prof. P. P. Murthy} \\
			
			
		\end{center}
		
		\date{}
		
		
		
		\newpage
		\hskip 4cm
	
	\newline
	\hskip 4.1cm \textbf{{\Large DECLERATION}\begin{figure}[h]
			\centering
			\includegraphics[width=0.3\linewidth]{"../../../Downloads/Web capture_28-6-2023_175856_www.bing.com"}
	\end{figure}}
	\vskip 1.5cm
	\author{\textbf{\centering Department of Mathematics\\ \hskip 4cm Guru Ghasidas Vishwavidyalaya}}
	\newline \vskip 1cm I hereby declare that the entire work presented in the project work on \textbf{\small RSA vs ECC : A COMPARTIVE ANALYSIS} submitted for partial fulfillment of M.sc Mathematics has been performed in the Department of Mathematics Guru Ghasidas Vishwavidalaya, Bilaspur under the supervision of Prof. P.P. Murthy.
	\newline The work presented in this disseration is orignal and will remain intellectual property of Department of Mathematics, Guru Ghasidas Vishwavidalaya, Bilaspur(C.G).
	\begin{flushright}
		\vskip 2cm
		Ujjwal Kumar Garg\\M.sc. Mathematics\\Roll No. 21075153
	\end{flushright}
		\newpage
		\hskip 4.1cm \textbf{{\Large CERTIFICATE}\begin{figure}[h]
			\centering
			\includegraphics[width=0.3\linewidth]{"../../../Downloads/Web capture_28-6-2023_175856_www.bing.com"}
	\end{figure}}
	\vskip 1.5cm
	\newline
	
	\author{\textbf{\centering Department of Mathematics\\ \hskip 4cm Guru Ghasidas Vishwavidyalaya}}
	\newline \vskip 1.5cm This is to certify that the dissertation entitled “\textbf{\small RSA vs ECC : A COMPARTIVE ANALYSIS}” is based on a part of research work carried out by \textbf{Ujjwal Kumar Garg} under my guidance and supervision at Guru Ghasidas Vishwavidyalaya, Bilaspur,(C.G), 495009
	\begin{flushright}
		..........\\Prof. P. P. Murthy\\Supervisor\\
		\vskip 2.25cm
		...........\\Dr. J. P. Jaiswal\\H.O.D
	\end{flushright}
	\begin{flushleft}
		Date: ..........
	\end{flushleft}
		\newpage
		\hskip 3.1cm \textbf{{\Large ACKNOWLEDGEMENT}}
		\vskip 1.1cm
			In the successful completion of this project, many people have bestowed upon me with their blessings and heartfelt support.
		 This time, I am utilizing this opportunity  to thank all the people who have been concerned with this project.
		I would thank God for allowing me to complete this project successfully.
		 I would like to thank Head of Department Dr. J. P. Jaiswal, whose valuable support help througout this project. I express my gratidue to my supervisour Prof. P. P. Murthy  for his stimulated discussion, crtical analysis and support from time to time. Last but not least I would like to thank my friends and research scholars of department who have helped me with their valuable suggestions and guidance has been very helpful in various phases of the project.
		\begin{flushleft}
			\vskip 2.5cm
			\textbf{{\Large 	Ujjwal Kumar Garg}}
		\end{flushleft}
		\newpage
	
		\tableofcontents
		\newpage
		
		
		
		\pagenumbering{arabic}
		\newpage
	\section{History}
	There have been \textbf{Three well-defined Phases} in the history of cryptology. \vskip .25cm The \textbf{First phase} was the period of manual cryptography, starting with the origins of the subject in antiquity and continuing through World War I. This phase cryptography was limited by the complexity of what a code clerk could reasonably do aided by simple mnemonic devices. As a result, ciphers were limited to at most a few pages in size, i.e., to only a few thousands of characters.M General principles for both cryptography and cryptanalysis were known, but the security that could be achieved was always limited by what could be done manually. Most systems could be cryptanalyzed, therefore, given sufficient ciphertext and effort. One way to think of this phase is that any cryptography scheme devised during those two millennia could have equally well been used by the ancients if they had known of it.
	\vskip 0.35cm The \textbf{Second Phase},  the mechanization of cryptography, began shortly after World War I and continues even today. The applicable technology involved either telephone and telegraph communications (employing punched paper tape, telephone switches, and relays) or calculating machines such as the Brunsvigas, Marchants, Facits, and Friedens (employing gears, sprockets, ratchets, pawls, and cams). This resulted in the rotor machines used by all participants in World War II.
	\vskip .35cm The \textbf{Third Phase}, dating only to the last two decades of the 20th century, marked the most radical change of all—the dramatic extension of cryptology to the information age: digital signatures, authentication, shared or distributed capabilities to exercise cryptologic functions, and so on. It is tempting to equate this phase with the appearance of public-key cryptography, but that is too narrow a view. Cryptology’s Third Phase was the inevitable consequence of having to devise ways for electronic information to perform all of the functions that had historically been done .
	\newpage
	\subsection{1920 B.C Egypt-Hieroglyph}
	It was used as early as 1900 BC in ancient Egypt. During these time the
	Egyptians would create a code using Hieroglyphics by switchings the order
	of them and only the people who knew the order could the translate the message
	\newline
	In this Hieroglyphic alphabets design like as sanke means the woerd "j" symbols likes the waves is stand for "n" and the same as the meaning of the all the symbol.
	\begin{figure}[h]
		\centering
		\includegraphics[width=0.9\linewidth]{"../../../Downloads/Web capture_29-6-2023_1049_www.bing.com"}
		\caption{}
		\label{Hieroglyph}
	\end{figure}
	\subsection{1500 B.C Babylonian}
	The lenticular clay tablet was used to help scribes learn to write the sumerian and Akkadian languages using the triangle-like cuneiform script.To learn a word or sign,the teacher would write the form on the obverse,and the student would then repeat the excerise on the revrse.Such elementry excerise were often completed on the tablets that were smalland round,easily fitting into the plam of hand
	\newline  On this tablet,the name of the deity Urash was copied six times.(Additional signs seem to be present on the reverse but are too damaged to read).Two signs used to write this name:the first star-like sign on the left is a sign thst was used the indicate the name ofa for Urash ,the name of diety.
	cuneiform writing,therefore,requried the mastery of sevral hundred sign and their diffrent meaning.A Babyloian cuneiform tablet ,dating from about 1500 BC,contains an encrypted recipe for making pottery glaze.This example more likely represents the occurence of ciphers in this part of the world as this would be fairly trivial.
	\begin{figure}[h]
		\centering
		\includegraphics[width=0.7\linewidth=1cm]{"../../../Downloads/Web capture_29-6-2023_1348_www.bing.com"}
		\caption{}
		\label{Babylonian}
	\end{figure}
	
	\subsection{58 B.C. Julius-Ceaser}
	The first important use of cryptography is the reloctation code.This was described by the greek writer polyibus as a substitution technique but his military use was developed by the Roman Emperor Julius ceasar(58BC).Caesar communicated with his commanders through this encryption method. messages are replacing the letter in the text with one of the three position to the right. For example,the word ENDER was changed to HQGHU.
	\subsection{5 B.C. Spatra}
	In the 5th century BC,the first displacment system was introduced by a method developed by sparta.For this reason,the first nation to use cryptography in military communicartrion is referred to as sparta.The developed device consisted of a wooden roller of a certain thickness and a papryus or a thin, leather band that was bent around the cylinder. the hidden message was written over the roll along the roll,and then the strip was unwound ffrom the cylinder and transmitted to the desired traget. Here ,the diameter of the cylinder served as a key to the encryption and resoulation of the text.
	
		\subsection{History of Enigma}
	
	The \textbf{History of the Enigma} starts \textbf{around 1915}, with the invention of the rotor-based cipher machine. As usual in history, the rotor machine was invented more or less simultaneously in different parts of the world. In 1917 there were inventions from Edward Hebern in the USA, Arvid Damm in Sweden, Hugo Koch in The Netherlands and Arthur Scherbius in Germany
	
	\newline
	\vskip .2cm
	An Enigma machine is a famous encryption machine used by the Germans during (WWII) to transmit coded messages. An Enigma machine allows for billions and billions of ways to encode a message, making it incredibly difficult for other nations to crack German codes during the war — for a time the code seemed unbreakable.
	The Enigma machine was invented by Arthur Scherbius in 1918, right at the end of World War I. After several years of improving his invention, the first machine saw the light of day in 1923. A year earlier he had secured the rights to patent NL10700 of Dutch inventor Hugo Koch for a similar device.
	\newline
	\vskip .2cm
	 The Enigma machine, which combined electrical and mechanical components, was descended from a number of designs that were submitted for patent as early as 1918 in Germany and were produced commercially beginning in the early 1920s. Looking rather like a typewriter, it was battery-powered and highly portable. In addition to a keyboard, the device had a lamp board consisting of 26 stenciled letters, each with a small lightbulb behind it. As a cipher clerk typed a message on the keyboard in plain German, letters were illuminated one by one on the lamp board. An assistant recorded the letters by hand to form the enciphered message, which was then transmitted in Morse Code.Each bulb in the lamp board was electrically connected to a letter on the keyboard, but the wiring passed via a number of rotating wheels, with the result that the connections were always changing as the wheels moved. Thus, typing the same letter at the keyboard, such as AAAA..., would produce a stream of changing letters at the lamp board, such as WMEV…. It was this ever-changing pattern of connections that made Enigma extremely hard to break.
	
	\begin{figure}[h]
		\centering
		\includegraphics[width=0.7\linewidth]{"../../../Downloads/Web capture_1-7-2023_231025_www.bing.com"}
		\caption{}
		\label{Enigma}
	\end{figure}
	
	
	\newline
	 The earliest success against the German military Enigma was by the Polish Cipher Bureau. In the winter of 1932–33, Polish mathematician Marian Rejewski deduced the pattern of wiring inside the three rotating wheels of the Enigma machine. (Rejewski was helped by photographs, received from the French secret service, showing pages of an Enigma operating manual for September and October 1932.) Before an Enigma operator began enciphering a message, he set Enigma’s three wheels (four in models used by the German navy) to various starting positions that were also known to the intended recipient. In a major breakthrough, Rejewski invented a method for finding out, from each intercepted German transmission, the positions in which the wheels had started at the beginning of the message. In consequence, Poland was able to read encrypted German messages from 1933 to 1939. In the summer of 1939 Poland turned over everything—including information about Rejewski’s Bomba, a machine he devised in 1938 for breaking Enigma messages—to Britain and France. In May 1940, however, a radical change to the Enigma system eliminated the loophole that Rejewski had exploited to discover the starting positions of the wheels.
	
	\vskip 2.12cm
	\textit{“Cryptography without system integrity is like investing in an armored car to carry money between a customer living in a cardboard box and a person doing business on a park bench.”} — \textbf{Gene Spafford}
	
	\newpage
	\section{Cryptography}
	
	
	Cryptography is the study of secure communications techniques that allow only the sender and intended recipient of a message to view its contents. The term is derived from the Greek word kryptos, which means hidden. It is closely associated to encryption, which is the act of scrambling ordinary text into what's known as ciphertext and then back again upon arrival. In addition, cryptography also covers the obfuscation of information in images using techniques such as microdots or merging. Ancient Egyptians were known to use these methods in complex hieroglyphics, and Roman Emperor Julius Caesar is credited with using one of the first modern ciphers.
	\subsection{Definitions}
	
	\vskip 0.4cm
	Cryptography is the process of hidding or coding of information so that only the person a message was intended for can read it. The art of cryptography has been used to code messages for thousands of years and continues to be used in bank cards, computer passwords, and ecommerce.
	\vskip .4cm
	A common cryptography definition is the practice of coding information to ensure only the person that a message was written for can read and process the information. This cybersecurity practice, also known as cryptology, combines various disciplines like computer science, engineering, and mathematics to create complex codes that hide the true meaning of a message.
	\vskip .4cm
	Cryptography is the study and practice of techniques for secure communication in the presence of third parties called adversaries. It deals with developing and analyzing protocols which prevents malicious third parties from retrieving information being shared between two entities thereby following the various aspects of information security
	\vskip 2cm
	\textit{“A cryptographic system should be secure even if everything about the system, except the key, is public knowledge.”} — \textbf{Auguste Kerckhoffs}
	
	\newpage
	\subsection{Importance of Cryptography}
	\vskip .3cm
	Cryptography is an essential information security tool. It provides the four most basic services of information security
	\vskip .3cm 
	\begin{itemize}
		\item\textbf{ Confidentiality }: Encryption technique can guard the information and communication from unauthorized revelation and access of information.
		\item\textbf{ Authentication }: The cryptographic techniques such as MAC and digital signatures can protect information against spoofing and forgeries.
		\item\textbf{ Data Integrity} : The cryptographic hash functions are playing vital role in assuring the users about the data integrity.
		\item\textbf{ Non-repudiation} : The digital signature provides the non-repudiation service to guard against the dispute that may arise due to denial of passing message by the sender.
	\end{itemize}
	All these fundamental services offered by cryptography has enabled the conduct of business over the networks using the computer systems in extremely efficient and effective manner.
	\vskip 3.12cm
	\textit{“Cryptography without system integrity is like investing in an armored car to carry money between a customer living in a cardboard box and a person doing business on a park bench.”} — \textbf{Gene Spafford}
	
	\newpage
	\subsection{Terminology;}
	\begin{itemize}
		\item \textbf{Plain Text: }It is usually a ordinary readble text.
		
		\item \textbf{Cipher Text:} Cipher text is encrypted text tranform from plain text using encryption algorithm.
		
		\item \textbf{Encryption:} Encryption is a process which transforms the original information into an unrecognizable form. This new form of the message is entirely different from the original message
		
		\item \textbf{Decryption:} Decryption is the process of converting an encrypted message back to its original format that is plain text.
		
		\item \textbf{Key:} A key is a string of characters used within an encryption algorithm for altering data so taht apperas random.
	
	\end{itemize}
	\vskip .8cm
		\subsection{Types of Cryptography}
	{\large $1$.} {\large \textbf{Symmetric Key Cryptography}}
	\vskip .810cm
	Symmetric key encryption, also called private key cryptography, is an encryption method where only one key is used to encrypt and decrypt messages. This method is commonly used in banking and data storage applications to prevent fraudulent charges and identity theft as well as protect stored data.
	\newline A sender and their designated recipients have identical copies of the key, which is kept secret to prevent outsiders from decrypting their messages. The sender uses this key to encrypt their messages through an encryption algorithm, called a cipher, which converts plaintext to ciphertext. The designated recipients then use the same key to decrypt the messages by converting the ciphertext back to plaintext.
	
	\begin{figure}[h]
		\centering
		\includegraphics[width=0.7\linewidth]{"../../../Downloads/Web capture_2-7-2023_1243_www.bing.com"}
		\caption{}
		\label{Symmetric Key}
	\end{figure}
	
	\vskip .3cm
	
	\textbf{Categories of Symmetric Key Encryption}
	\begin{itemize}
		\vskip 1.1cm
		\item \textbf{Data Encryption Standard (DES):} It was developed in the early 1970s and is considered a legacy encryption algorithm. This block cipher used 56-bit keys and encrypted block sizes of 64 bits. Due to its short key length, the encryption standard was not very secure. However, it played a vital role in the advancement of cryptography. Because the US National Security Agency (NSA) participated in DES’s development, many academics were skeptical. This skepticism caused a surge in cryptography research, which led to the modern understanding of block ciphers.
		\vskip 1cm
		\item \textbf{Advanced Encryption Standard (AES:}) It also known as Rijndael, supersedes DES. AES is an encryption standard used by the US government to encrypt classified information. It is also popular with companies such as Google, Mozilla, and Microsoft. It is a family of block ciphers developed by the Belgian cryptographers Vincent Rijmen and Joan Daemen. The AES family can handle block sizes and encryption key sizes of 128, 160, 192, 224, and 256 bits. Officially, only 128-, 192-, and 256-bit key sizes and a 128-bit block size are specified in the encryption standard.
		
	\end{itemize}
\vskip .5cm
\textit{“Every secret creates a potential failure point.”} — \textbf{Bruce Schneier}
	\newpage
	{\large $2$}.\textbf{{\large  Asymmetric Key Cryptography}}
	\vskip .8cm
	Asymmetric cryptography is also called public key cryptography because it consists of two keys one is public key and second is private key.One key is used for encryption and  the other key  should be used for decryption. A public key is a cryptographic key that can be used by any person to encrypt a message so that it can only be decrypted by the intended recipient with their private key. A private key -- also known as a secret key - is shared only with key's initiator.here is no other key can decrypt the message and not even the initial key used for encryption. The style of the design is that every communicating party needs only a key pair for communicating with any number of other communicating parties.\\
	Asymmetric cryptography is scalable for use in high and ever expanding environments where data are generally exchanged between different communication partners. Asymmetric cryptography is used to exchange the secret key to prepare for using symmetric cryptography to encrypt information.
	\newline In the case of a key exchange, one party produce the secret key and encrypts it with the public key of the recipient. The recipient can decrypt it with their private key. The remaining communication would be completed with the secret key being the encryption key. Asymmetric encryption is used in key exchange, email security, Web security, and some encryption systems that needed key exchange over the public network.
	\begin{figure}[h]
		\centering
		\includegraphics[width=0.7\linewidth]{"../../../Downloads/Web capture_2-7-2023_11716_www.bing.com"}
	\end{figure}
	
		
	
	\newpage
	\section{RSA}
	RSA is considered as the first real life and practical 
	asymmetric-key cryptosystem. It becomes de facto standard 
	for public-key cryptography. Its security lies with integer 
	factorization problem. RSA’s decryption process is not 
	efficient as its encryption process. Many researchers have 
	proposed to improve the efficiency of RSA’s decryption using 
	Chinese Remainder Theorem (CRT). 
	\newline \vskip .2cm
	RSA algorithm was presented by a group of a 
	security researchers, Ronald Rivest, Adi Shamir, and 
	Leonard Adleman. The term RSA was derived from the 
	names of the three developers of this algorithm. Till 
	now, RSA is most widely used as a public key 
	cryptographic algorithm. 
	
	\subsection{Algorithim}
	\vskip .5cm
	
	The idea of RSA is based on the fact that it is difficult to factorize a large integer. The public key consists of two numbers where one number is a multiplication of two large prime numbers. And private key is also derived from the same two prime numbers. So if somebody can factorize the large number, the private key is compromised. Therefore encryption strength totally lies on the key size and if we double or triple the key size, the strength of encryption increases exponentially.
	
	\vskip 1cm 
	\textbf{Key Generation}
	\begin{itemize}
		\item Step I. Select p and q, where p and q both are prime numbers, p not equal to q
		\item Step II. Calculate n = pq
		\item Step III. Calculate $\phi$(n) = $(p - 1)(q - 1)$
		\item Step IV. Select integer e such that $1$< e< $\phi$(n) ; gcd($\phi$(n),e) = $1$
		\item Step V. Calculate d ;  ed = \mod ($\phi$(n)) 
		\item Step VI. Public key pair $=$ (n , e )
		\item Step VII. Private key pair $=$ (n , d) 
	\end{itemize}
	\newpage
	\textbf{Encryption}
	\vskip.2cm
	\begin{itemize}
		\item Step I. Plaintext: M < n, where M is a plain text
		\item Step II. Ciphertext: $C = M^{e}$ $\mod$ n
		
	\end{itemize}
	\vskip.2cm
	\textbf{Decryption}
	\vskip.2cm
	\begin{itemize}
		\item Step I. Ciphertext: C
		\item Step II. Plaintext M $ = $ $C^{d}$ $\mod$ n
	\end{itemize}
\subsection{Security of RSA}

There are three main approaches of attacking RSA algorithm.

\vskip .5cm
\textbf{Brute force key search} (infeasible given size of numbers) As explained
before, involves trying all possible private keys. Best defense is using large keys.

\vskip .5cm
Mathematical attacks (based on difficulty of computing $\phi$(N), by
factoring modulus N) There are several approaches, all equivalent in efect
to factoring the product of two primes. Some of them are given as:\\ 

\item  factor $N$ = $p*q$, hence find $\phi$(N) and then d
\item determine $\phi$(N) directly and find d
\item find d directly\\ 

The possible defense would be using large keys and also choosing large
numbers for p and q, which should differ only by a few bits and are also on
the order of magnitude 1075 to 10100.

	\newpage
	\subsection{Encryption and Decryption of plaintext}
	\begin{itemize}
		\item Choose two prime numbers p and q where p = $7$ and q = $17$
		\item Now calculate n = $p*q$ = $7*17$=$119$
		\item Calculate $\phi$(n) = (p - 1)(q - 1) = $(7 - 1)(17 - 1) $\\ $6*16 = 96$
		
		\item Since the factors of $96$ are $2*2*2*2*2*3$ therefore e select such that none of the factor of e as $2$ and $3$ 
		\item Let us choose e = $5$
		\item Now we have a public key pair = (n , e) = $(5,119)$
		\item Calcuation of d\\ ed=mod $\phi$(n)=$1$ so we get\\ d = $77$
		\item private key pair = (n , d) = (119 , 77)
	\end{itemize}
	\subsection{Encryption of plaintext:}
	\begin{itemize}
		\item Let m is our plaintext and m = $10$
		\item the public key for encryption of plaintext is \\ c = $m^e \mod(n)$ \\ c=$ 10^5 \mod(119)$\\ c = $100000$ mod$(119)$\\ c = $40$\\ c = $40$ is our ciphertext
	\end{itemize}
	\subsection{Decryption of Ciphertext:}
	\begin{itemize}
		\item the private key for decryption of plaintext is\\ m = $c^d \mod(n)$
		\item m = $40^{77} \mod(119)$\\ m = $10$
	\end{itemize}
	
	\newpage
	\section{ECC}
	\subsection{Histroy of ECC}
	
The properties and function of elliptic curves in mathematics have been studied for more than $1500$ years. Their use within cryptography was first proposed in $1985$, separtly by Neal Koblitz from the university of Washington and Victor Miller at IBM.
\\ They are the elliptic curves analogues if scheme based on the discrete logrith problem , where the underlying group is the group of points on an elliptic curve defined over a finite field.
	
	
	
	
\subsection{Elliptic Curves Cryptography}
	\vskip .3cm
	The principal attraction of ECC compared to RSA is that 
	it offers equal security for a far smaller key size, thereby 
	reducing processing overhead. The addition operation in 
	ECC is the counterpart of modular multiplication in RSA, 
	and multiple addition is the counterpart of modular 
	exponentiation. To form a cryptographic system using 
	elliptic curves, we need to find a “hard problem”. All 
	systems rely on the difficulty of a mathematical problem 
	for their security. To explain the concept of difficult 
	mathematical problem, the notion of an algorithm is 
	required. To analyze how long an algorithm takes, 
	computer scientists introduced the idea of polynomial time 
	algorithms and exponential time algorithms. An algorithm 
	runs quickly if it is polynomial time algorithm, and slowly 
	if it is exponential time algorithm. Therefore, easy 
	problems equate with polynomial time algorithms, and 
	difficult problems equate with exponential time 
	algorithms. When looking for a mathematical problem on 
	which to base a public key cryptographic system, 
	cryptographers search for a problem for which the fastest 
	algorithm takes exponential time. The longer it takes to 
	compute the best algorithm for a problem, the more secure 
	a public key cryptosystem based on that problem will be
	\newline
	\vskip .2cm
	The Elliptic Curve Cryptosystem, whose security 
	rests on the discrete logarithm problem over the points on 
	the elliptic curve. The main attraction of ECC over RSA 
	and DSA is that the best known algorithm for solving the 
	underlying hard mathematical problem in ECC (the elliptic 
	curve discrete logarithm problem (ECDLP) takes full 
	exponential time. 
	
	\newpage
	\textbf{$\bullet$ Definition:}
	\vskip .2cm 
	An elliptic curve is the set of points that satisfy a specific mathematical  equation.\\For ECC, we are concerned with a 
	restricted form of elliptic curve that is defined over a finite 
	field. Of particular interest for cryptography is what is 
	referred to as the elliptic group mod p, where p is a prime 
	number. This is defined as follows. Choose two 
	nonnegative integers, a and b, less than p that satisfy:\\ \vskip .4cm
	
	$$4a^{3} + 27b^{2}\neq 0 \mod p  $$ \\ \vskip .2cm Then $E_{q}(a,b)$ denotes the elliptic group mod p whose 
	elements (x, y) are pairs of nonnegative integers less than p satisfing :
	\newline
	\vskip .2cm
	
	$$y^{2} = x^{3}+ax+b \modp$$
	
	$\bullet$ The elliptic curve discrete logarithm problem can be stated 
	as follows.
	\newline Fix a prime p and an elliptic curve. $$Q= xP$$
	where xP represents the point P on elliptic curve added to 
	itself x times. Then the elliptic curve discrete logarithm 
	problem is to determine x given P and Q. It is relatively 
	easy to calculate Q given x and P, but it is very hard to determine x given Q and P.
	\newline
	\begin{figure}[h]
		\centering
		\includegraphics[width=0.7\linewidth]{"../../../Downloads/Web capture_2-7-2023_21134_www.bing.com"}
		\caption{}
		\label{Elliptic Curve}
	\end{figure}


	
	
	\newpage
	\subsection{Algorithm }
	\vskip .6cm
	\textbf{$\bullet$ Global Public Elements: }
	\vskip .2cm 
	\begin{itemize}
		\item Step I. $E_{q}(a, b)$ elliptic curve with parameters a, b, and q, 
		where q is a prime or integer of the form $2^{m}$. 
		\item Step II. G point on elliptic curve whose order is large value n 
	\end{itemize}
	\vskip .2cm
	\textbf{$\bullet$ User A Key Generation:}
	\begin{itemize}
		\item Step I. Select private key $n_{A} ;    n_{A}<n$
		\item Step II. Calculate public key $P_{A}$
		\item Step III. $P_{A} = n_{A}*G$
	\end{itemize}
	\vskip .2cm
	\textbf{$\bullet$ User B Key Generation:}
	\begin{itemize}
		\item  Select Private key  n_{B} ;   n_{B}<n
		\item  Calculate Public key  $P_{B}$
		\item  $P_{B} = n_{B}*G$
	\end{itemize}
	\textbf{$\bullet$ Calculation of Secret Key by User A: }
	\vskip .2cm
	
	
	\begin{itemize}
		\item Step I. $K = n_{A}*P_{B}$
	\end{itemize}
	\textbf{$\bullet$ Calculation of Secret Key by User B: }
	\vskip .2cm
	
	\begin{itemize}
		\item Step I. $K = n_{B}P_{A}$
	\end{itemize}
	\textbf{$\bullet$ Encryption by A using B Public Key:}
	\vskip .2cm
	\begin{itemize}
		\item A chooses message $P_{m}$ and a random positive 
		integer k
		\item Ciphertext  $C_{M} = {KG, P_{M}+kP_{B}}$
	\end{itemize}
\vskip .2cm
\textbf{$\bullet$ Decryption by B using his own Private Key:}

\begin{itemize}
	\item Ciphertext: $C_{m}$
	\item $ P_{M}+kP_{B}-kG*n_{B} = P_{M}$
\end{itemize}

	\begin{figure}[h]
	\centering
	\includegraphics[width=0.7\linewidth]{"../../../Downloads/Web capture_2-7-2023_23114_www.bing.com"}
	\caption{elliptic curve}
	\label{P+Q}
\end{figure}



	\newpage
	\begin{figure}[h]
		\centering
		\includegraphics[width=0.7\linewidth]{"../../../Downloads/Web capture_2-7-2023_24125_www.bing.com"}
		\caption{}
		\label{nP}
	\end{figure}
	
	
	\textbf{$\bullet$ Calculation of P+Q:}
	
\begin{itemize}
	\item Let $P = (x_{1},y_{1})$ and$ Q = (x_{2},y_{2})$
	\item And let $P+Q = (x_{3},y_{3})$
	\item $x_{3} = $\lambda$^{2} - x_{1}-x_{2}\modp$
	\item and $y_{3} = $\lambda$(x_{1}-x_{3})-y_{1}modp$ 
\hskip .4cm	\item  Where $\lambda$ = $\dfrac{y_{2}-y_{1}}{x_{2}-x_{1}}$ \modp$
\end{itemize}

\vskip 2cm

 \textbf{$\bullet$ Calculation of $2P$:}
 \begin{itemize}
 	\item Let P = Q then 2P equal to
 	\item $\lambda$ = $\dfrac{3x_{1}^{2} + a}{2y_{1}}$ \modp$
 	\item$ x_{3} = $\lambda$^{2}-x_{1}-x_{2}\modp$
 	\item $y_{3}$ = $\lambda$(a-x_{3})-y \modp$
 	
 \end{itemize}

\vskip .4cm
  \subsection{Problem on ECC}
  \vskip .2cm
\textbf{	$\bullet$ Global parameter of ECC are:}
\newline
\vskip .3cm
Here prime number $p=11, a=1, b=1$ for encoding and decoding of message in ellptic curve. Based on global parameters, the ellptic curve equations become: \\ \vskip .1cm $$y^{2}\mod11 = (x^{3}+x+1)\mod11$$
	\vskip .5cm
	\begin{tabular}{|c|c|c|}
		\hline
		GF(11)	& $y^{2}mod11$ & $x^{3}+2x+1mod11$ \\
		\hline
		0	& 0 & 1 \\
		\hline
		1	& 1 & 3 \\
		\hline
		2	&  4& 0 \\
		\hline
		3	&  9& 3 \\
		\hline
		4	& 5 & 3 \\
		\hline
		5	& 3 & 10 \\
		\hline
		6	& 3 & 3 \\
		\hline
		7	& 5 & 10 \\
		\hline
		8	& 9 & 4 \\
		\hline
		9	& 4 & 2 \\
		\hline
		10	& 1 &10 \\
		\hline
	\end{tabular}
	
	
	
	
	
	
	
	
	
	
	\vskip .5cm
	\begin{itemize}
		\item Step$:1$ Encoded a plain text message as a point on the curve \\ Lets consider the point to be encoded plain text message on the curve  $$M \in E_{11}(1,1) is (4,6)$$
		
		\item Step$:2$ Eastablish the Public key and Privte key \\ Chose a generator point $$G\in E_{11}(1,1)$$ let G is $$(1,5)\in E_{11}(1,1)$$ \\ \vskip .2cm Select a private key $n=2$ \\ \vskip .2cm Compute the Public key as $P_{A} = nG$
		
		\item Let nG equal to $(x_{3},y_{3})$\mod11 
		\item  $as $n=2$  and  $G = (1,5)$
		\item $P_{A} = 2G = G+G = (1,5)+(1,5)$
		\item Let $x_{1}= x_{2}=1$ and $y_{1}= y_{2}=5$ 
		\item  $\lambda$ = $\dfrac{3x^{2}+a}{2y_{1}}$ \mod11 
		\item $\dfrac{3*1^{2}+1}{2*5}$\mod11=7
		\item 	$x_{3}$ = $\lambda$^{2}$-x_{1}-x_{2}$\mod11 $= 7^{2}-1-1\mod11 = 3$
		\item y_{3}= $\lambda$(a-x_{3}) - $y\mod11 = 7(1-3) - 5\mod11 = 3$
		
		\item y_{3}= $\lambda$(a-x_{3}) - $y\mod11 = 7(1-3) - 5\mod11 = 3$
		\item now we have $(x_{3},y_{3}) = (3,3)$
		
	\end{itemize}
	
	\vskip .5cm
	\textbf{$\bullet$ Step:3 Encrypt the message using Public key}
	
		\begin{itemize}
		\item 	$C = [kG, M+kP_{A}]$  where k is a random number
		\item $C = [C_{1},C_{2}]$
		\item \vskip .2cm Let $k = 2$ \\ \vskip .2cm $C= [2(1,5), (4,6)+2(3,3)]$ \\ \vskip .2cm $C = [(1,5)+(1,5), (4,6)+(3,3)+(3,3)]$ \\ \vskip .2cm $C = [(3,3),(4,6)+(3,3)+(3,3)]$ \\ \vskip .2cm $C = [(3,3), (4,6)+(6,5)]$ \\ \vskip .2cm $C = [(3,3),(4,5)]$ \\ \vskip .2cm $C_{1} = (3,3)$ and $C_{2} = (4,5)$
	\end{itemize}

	
	

	
	\newpage 
	\textbf{$\bullet$ Step:4 Decrypt using private key:}
	\vskip .3cm
		$M = C_{2}-[nC_{1}]$ \\ \vskip .2cm $M = (4,5) - [2(3,3)]$ \\ \vskip 0.2cm $M = (4,5) - [(3,3),(3,3)]$ \\ \vskip .2cm $M = (4,5) - (6,5)$ \\ \vskip .2cm $M = (4,5)+ (6,-5)$ \\ \vskip.2cm $\lambda$ = $\dfrac{y_{2}- y_{1}}{x_{2}-x_{1}}$\mod 11 = $\dfrac{-5-5}{6-4}$\mod11$ = -5\mod11 = 6 \\ \vskip .2cm $x_{3}$ = $\lambda$^{2}-x_{1}-x_{2}\mod11 $= 6^{2}-4-6\mod11 = 26\mod11 = 4$ \\ \vskip .2cm $y_{3}$ = $\lambda$(x_{1}-x_{3}) $- y_{1}$\mod11 = 6(4-4)-5$\mod11 = 6$ \\ \vskip .2cm $(x_{3},y_{3}) = (4,6)$
	
		\newpage
		\subsection{Analysis of RSA vs ECC}
	\vskip .6cm
		
	$\bullet$ \hskip .6cm $8$ bits – Encryption and  Decryption
		\vskip .7cm
		\begin{tabular}{|c|c|c|c|c|}
			\hline
			Security \\ bit level	& ECC 
			Enc. 
			Time & RSA Enc. 
			Time & ECC Dec. 
			Time & RSA Dec. 
			Time \\
			\hline
			80	& 0.4885  & 0.0307 & 1.3267  &0.7543  \\
			\hline
			112	& 2.2030 & 0.0299  & 1.5863 &  2.7075 \\
			\hline
			128	& 3.8763 & 0.0305 & 1.7690 & 6.9409 \\
			\hline
			144	& 4.7266 &  0.0489 & 2.0022  &  13.6472 \\
			\hline
		\end{tabular}
	
	\vskip .6cm
	$\bullet$ \hskip .3cm $64$ bits – Encryption and Decryption
	\vskip .7cm
	\begin{tabular}{|c|c|c|c|c|}
		\hline
		Security\\ bit level	& ECC 
		Enc. 
		Time & RSA Enc. 
		Time & ECC Dec. 
		Time & RSA Dec. 
		Time \\
		\hline
		80	& 2.1685 & 0.1366 & 5.9099 & 5.5372 \\
		\hline
		112	& 9.9855 & 0.1635 & 6.9333 & 20.4108 \\
		\hline
		128	& 15.0882 & 0.1672 & 7.3584 & 46.4782 \\
		\hline
		144	& 20.2308 & .01385 & 8.4785 & 77.7642 \\
		\hline
	\end{tabular}
	
	\vskip .6cm
	$\bullet$ \hskip .3cm $256$ bits – Encryption and Decryption
	\vskip .7cm
	\begin{tabular}{|c|c|c|c|c|}
		\hline
		Security \\ bit level	& ECC 
		Enc. 
		Time & RSA Enc. 
		Time & ECC Dec. 
		Time & RSA Dec. 
		Time \\
		\hline
		80	& 7.9240 & 0.5596 & 22.8851 & 19.3177 \\
		\hline
		112	& 39.7008 & 0.5815 & 26.3331 &102.0337  \\
		\hline
		128	& 58.4386 & 0.5611  & 27.4060 &209.6086  \\
		\hline
		144	& 77.5034 & 0.5718 & 32.1522  & 311.0649 \\
		\hline
	\end{tabular}
	
	
	
	\newpage
	\begin{itemize}
		\item 	x = [80,112,128,144]
		\item y=[.4885,2.2030,3.8963,4.9266]
		\item z=[0.0307,0.0299,0.0305,0.0489]
		\item 	plt.plot(x,y , linewidth = 3)
		\item plt.plot(x,z , linewidth =4)
		\item 	plt.scatter(x,y , marker="+" , linewidth=9)
		\item 	plt.scatter(x,z , marker = "." ,linewidth=9)
		\item plt.title("RSA vs ECC : Encryption time of 8 bits")
		\item 	plt.xlabel("RSA vs ECC : Ecryption Time of 8 bits")
		\item plt.ylabel("Time in second")
		\item plt.grid("true")
		\item 	plt.show()
	\end{itemize}
	plain Line = ECC, Point Line = RSA
\begin{figure}[h]
	\centering
	\includegraphics[width=0.8\linewidth]{"../../../Downloads/Web capture_16-7-2023_145136_localhost"}
\end{figure}


		
	
\newpage
\begin{itemize}
	\item x = [80,112,128,144]
	\item y1 = [1.3267,1.5863,1.7690,2.0022]
	\item z1 = [0.7543,2.7075,6.9409,13.6472]
	\item plt.plot(x,y1 , linewidth=3)
	\item plt.plot(x,z1 , linewidth=4)
	\item plt.scatter(x,y1,marker="+" , linewidth=9)
	\item plt.scatter(x,z1,marker=".",linewidth=9)
	\item plt.title("RSA vs ECC : Decryption Time of 8 bits")
	\item plt.xlabel("Security bit level")
	\item plt.ylabel("Time in second")
	\item plt.grid("true")
	\item plt.show()
\end{itemize}		
	
	\begin{figure}[h]
		\centering
		\includegraphics[width=0.9\linewidth]{"../../../Downloads/Web capture_16-7-2023_145358_localhost"}
	\end{figure}
	
	
		
	\newpage
		
	\begin{itemize}
		\item x = [80,112,128,144]
		\item y2 = [2.1685 ,9.9855 ,15.0882 ,20.2308]
		\item z2= [0.1366 ,0.1366 ,0.1672 ,0.1385]
	\item plt.plot(x,y2 , linewidth=3)
	\item plt.plot(x,z2 , linewidth=4)
	\item plt.scatter(x,y2,marker="+" , linewidth=9)
	\item plt.scatter(x,z2,marker=".",linewidth=9)
	\item plt.title("RSA vs ECC : Encryption Time of 64 bits")
	\item plt.xlabel("Security bit level")
	\item plt.ylabel("Time in second")
	\item plt.grid("true")
	\item plt.show()
		
	\end{itemize}
	
	\begin{figure}[h]
		\centering
		\includegraphics[width=0.9\linewidth]{"../../../Downloads/Web capture_16-7-2023_145532_localhost"}
	\end{figure}
	
	
	
	\newpage
	\begin{itemize}
		\item x = [80,112,128,144]
		\item y3 = [5.9099,6.9333,7.3584,8.4785]
		\item z3 = [5.5372,20.4108,46.4782,77.7642]
			\item plt.plot(x,y3 , linewidth=3)
		\item plt.plot(x,z3 , linewidth=4)
		\item plt.scatter(x,y3,marker="+" , linewidth=9)
		\item plt.scatter(x,z3,marker=".",linewidth=9)
		\item plt.title("RSA vs ECC : Decryption Time of 64 bits")
		\item plt.xlabel("Security bit level")
		\item plt.ylabel("Time in second")
		\item plt.grid("true")
		\item plt.show()
		
	\end{itemize}

	
\begin{figure}[h]
	\centering
	\includegraphics[width=0.9\linewidth]{"../../../Downloads/Web capture_16-7-2023_145637_localhost"}
\end{figure}

	
	
	\newpage
	\begin{itemize}
		\item x = [80,112,128,144]
		\item y4 = [7.9240,39.7008,58.4386,77.5034]
		\item y4 = [7.9240,39.7008,58.4386,77.5034]
			\item plt.plot(x,y4 , linewidth=3)
		\item plt.plot(x,z4 , linewidth=4)
		\item plt.scatter(x,y4,marker="+" , linewidth=9)
		\item plt.scatter(x,z4,marker=".",linewidth=9)
		\item plt.title("RSA vs ECC : Encryption Time of 256 bits")
		\item plt.xlabel("Security bit level")
		\item plt.ylabel("Time in second")
		\item plt.grid("true")
		\item plt.show()
	\end{itemize}
	
\begin{figure}[h]
	\centering
	\includegraphics[width=0.9\linewidth]{"../../../Downloads/Web capture_16-7-2023_14588_localhost"}
\end{figure}

	
	
	\newpage
	\begin{itemize}
		\item 	x = [80,112,128,144]
		\item y5 = [22.8851,26.3331,27.4060,32.1522]
		\item z5 = [19.3177,102.0337,209.6086,311.0649]
			\item plt.plot(x,y5 , linewidth=3)
		\item plt.plot(x,z5 , linewidth=4)
		\item plt.scatter(x,y5,marker="+" , linewidth=9)
		\item plt.scatter(x,z5,marker=".",linewidth=9)
		\item plt.title("RSA vs ECC : Decryption Time of 256 bits")
		\item plt.xlabel("Security bit level")
		\item plt.ylabel("Time in second")
		\item plt.grid("true")
		\item plt.show()
		
	\end{itemize}

\begin{figure}[h]
	\centering
	\includegraphics[width=0.9\linewidth]{"../../../Downloads/Web capture_16-7-2023_145913_localhost"}
\end{figure}

	

	
	
	
	
	
	
	\newpage
	
	\subsection{Advantages of ECC over RSA}
	\begin{itemize}
		\item ECC, it takes one$-$sixth the computational effort to provide the same level of cryptographic security that you get with $1024$ bit RSA and is $15$\\ \vskip .7cm
		
		
		\begin{tabular}{|c|c|c|} 
			\hline
			Symmetric Encryption\\ Key size\\ in bits	& RSA and DH key size & ECC key size \\
			\hline
			80	& 1024 & 160 \\
			\hline
			112	& 2048 & 224 \\
			\hline
			128	& 3072 & 256 \\
			\hline
			192	& 7680 &  384\\
			\hline
			256	& 15360 &  512\\
			\hline
		\end{tabular}
		\vskip .8cm
		\item Because of much smaller key sizes involved, ECC algorithm can be implemented on smartcards. Contactless smart cards work only with ECC
		\item ECC has become important for wireless sensor networks.
		\item ECC also serving as the standard mode of encryption that is used widely by various web applications.
		\item Poplur cryptocurrencies such as Bitcoin and Ethereum make use of the Elliptic Curve Digital Signature Algorithim particularly in signing transactions due to the security levels offered by ECC.
	\end{itemize}
	
	
	\newpage
	\section*{CONCLUSION}
	

	
	Security of the message is paramount during its transmission from one user to another user or system. A cryptographic technique provides a message security.
	 Symmetric-key  
	 cryptography is very good in providing security to the message but suffers from key distribution and management 	problems. We have compare both RSA and ECC based on the bits size on three sample input data of 8 bits, 64 bits, 256 bits by using python programming and we plotted the graph to show the comparison in time lapse on encyption and decryption.
	 
	
	

	\newpage
	
	\begin{thebibliography}{}
		
	\bibitem{} 	 Batina, L., Mentens, N., Sakiyama, K., Preneel, B., and Verbauwhed, I., Public- Key Cryptography on the Top of Niddle, https://www.researchgate.net/publication/221381449, (2007), 1831-1834.
	 
	\bibitem{} 	 J.A. Buchmann, Introduction to Cryptography, Second Edition, Springer, (2008)
	
	\bibitem{} Kapoor, V. , Abraham, V.S., and Singh, R., Elliptic curve cryptography, Ubiquity ( ACM, New York, NY, USA), 2008 (2008), 1-8.
	
	\bibitem{} Liestyuwati, B., Public Key Cryptography, {\textit Journal of physics (conference series)}, 1477 052062, (2020), 1-7.}

	
	\bibitem{} Sahinaslan.E,Sahinaslan.O,Cryptographic Methods and Development Stages Used Throughout History,AIP Conference Proceedings 2086, 030033 (2019),1-3
	
	\bibitem{}  Mahto, D., And Yadav, D.K., RSA and ECC : A comparative analysis, International Journal of applied engineering research, 12(19) (2017), 9053-9061.
	
	\bibitem{} 	 Mahto, D., And Yadav D. K., Performance Analysis of RSA and Elliptic Curve Cryptography, International Journal of Netw. Secur., 20(4) (2018), 625-635.
	
	\bibitem{}  Raju, GVS and Akbani, R., Elliptic curve cryptosystem and it's applications, SMC'03 Conference Proceedings (IEEE International Conference on Systems, Man and Cybernetics), 2 (2003), 1540-1543.
	
	\bibitem{}   The Code Book. The secret History of codes and codes-breaking,Simon Singh,(1999)
	
	\bibitem{}  Vigila, S.M.C., and Muneeswaran, k., Implementation of text based cryptosystem using elliptic curve cryptography, First  International Conference on Advanced Computing (IEEE), (2009), 82-85.
	
	\bibitem{}  W.stallings, Cryptography and Network security, 5th ed. Boston: Prentice Hall,(2011)
	

	

	
	
	
		
	\end{thebibliography}
	


	
	


	
	
	
	

	
	
	

	
\end{document}